\documentclass[a4paper,12pt]{article}

\usepackage{enumitem}
\usepackage[utf8]{inputenc}

% Blank line between paragraphs, no leading indent
\usepackage[parfill]{parskip}

% Automatic linking
\usepackage[pdftex,colorlinks=true]{hyperref}

% Rich selection of colours
\usepackage[usenames,dvipsnames]{xcolor}

% Strikeout text
\usepackage{soul}

% Embed images
\usepackage{graphicx}
\graphicspath{ {./images/} }
% Sane scaling
\usepackage[export]{adjustbox}

% Very slightly more relaxed spacing: http://en.wikibooks.org/wiki/LaTeX/Text_Formatting#Line_Spacing
\usepackage{setspace}
\setstretch{1.1}


%\usepackage{pifont} % ding
%\DeclareUnicodeCharacter {9320}{⑨}
% http://tex.stackexchange.com/questions/73560/error-unicode-char-u8%CF%86-not-set-up-for-use-with-latex
%
% Circled numbers
% http://tex.stackexchange.com/questions/7032/good-way-to-make-textcircled-numbers
% http://tex.stackexchange.com/questions/15334/getting-numbers-to-appear-in-circles
% I went with this in the end:
\newcommand{\maru}[1]{\raisebox{.5pt}{\textcircled{\raisebox{-.9pt} {#1}}}}
\newcommand{\vaultnine}{Vault \maru{9}}

% Italicise proper names for emphasis, and prevent breaking across lines
\newcommand{\propername}[1]{\mbox{\emph{#1}}}

% At the end of each instalment Anon wrote commentary, and sometimes I'll include survey responses.
% vfill: https://www.sharelatex.com/learn/Line_breaks_and_blank_spaces
\newenvironment{commentary}%
	{
		\vfill%
		\hrule%
		\begin{footnotesize}%
		\color{MidnightBlue}%
	}%
	{%
		\end{footnotesize}%
	}

% Easy embedding of images
% http://tex.stackexchange.com/questions/37790/scale-figure-to-a-percentage-of-textwidth
% http://tex.stackexchange.com/questions/41787/use-textwidth-for-image-width-only-when-it-outgrows-the-page
\newcommand{\img}[1]{%
	\includegraphics[max width=0.75\linewidth]{#1}%
}


\newcommand{\factionheading}[1]{
\parbox{\textwidth}{
        \vspace{2mm}
        \noindent
        \textcolor{MidnightBlue}{
{\large {#1}
        \vspace*{1mm}
        \hrule}
        \vspace*{3mm}
        \noindent
} } }




\begin{document}

\title{Fallout Gensokyo}
\author{By an Anonymous Fairy}
\date{2008}
\maketitle


%%%%%%%%%%%%%%%%%%%%%%%%%%%%%%%%%
\section{Preface}

This isn't mine, I'm just typesetting and editing.

Anon wrote this story on the \url{touhou-project.com} messageboards, taking
choose-your-own-adventure input from the readers. What accrued was a very fun
story with a great sense of world and supplementary media, in the form of music
and images.

Unfortunately, Anon suffered a hard drive crash and lost a lot of his raw
material, and things kinda petered out. Nevertheless, it's a very worthwhile
read for as far as it went (which is a long way).

I've tried to keep things pretty much verbatim from the original posts. Typos
and other trivial mistakes have been corrected. When Anon jumps out of story,
the text will switch to {\color{MidnightBlue} blue} and appear much like a
footnote. Embedded images are those given in the original posts, provided by
Anon to give some flavour to the writing.

With the announcement of Fallout 4, I felt like going over it again.  It's a
great story and I've always wanted to give it a proper edit job and typeset.
Well, it's about time. I hope you enjoy it as much as I did.

\begin{quote}
\emph{-- Furinkan}
\end{quote}



%%%%%%%%%%%%%%%%%%%%%%%%%%%%%%%%%%
%\newpage
\section{Introduction}

\img{reiuji-strangelove}

\begin{quote}
\emph{I don't want to set the world on fire;\\
I just want to start a flame in your heart.}
\end{quote}


War. War never changes. At the dawn of civilization, when our ancestors finally discovered themselves alone in this world, without predator or rival, they became faced with a quandary. For what now would they do with that which had ensured their survival in the wild, the ability to fashion weapons from their environment, as natural for them to use as the owl its talons and the wolf its fangs? So, with no other target, they would turn their spears on each other. At all stages of history, we can define progress of humanity not in the development of the sciences, medicine, arts, or culture, but by the ever increasing capacity to kill. Though the will to survive was itself not lost, mankind would develop ever an ever growing lexicon of terms that could be used to separate themselves from the other humans more deserving of death -- tribe, nation, class, race, creed. In the year 2077, in the ultimate culmination of arbitrary justifications for violent conflict, the humans of Earth, and the Lunarians, a label given to humans of her moon, Luna, would declare war on each other over a single flag, planted unassumingly over a century earlier. It is not agreed upon by Wasteland scholars who struck first, but what is known is that that retaliation on either side was total. Thus the Earth and her moon, after a momentary flash of light, were plunged into an era of unparalleled darkness.

But it was not, as some had predicted, the end of the world. Instead, the apocalypse was simply the prologue to another bloody chapter of history. Man had succeeded in destroying himself, yes, but there were yet other forces in this world ready to take up the reins of violence. Whether they were a product of mutation after the bombs fell, whether they had been bioengineered tools of war -- super-soldiers created to resist the deleterious effects of battlefield radiation -- or whether they had always existed and were just now no longer content to hang on the fringes of man's consciousness, the theories are as numerous as the forms they would take, and these meta-humans, known in the present day Gensokyo Wasteland as youkai, were all too willing to fill the gap left by humanity's destruction. Regeneration, the ability to go without food or water indefinitely, monstrous strength, and even more wondrous powers, flight and seeming magic -- like their mundane predecessors these beings would quickly learn to bend their abilities towards any who opposed them, including each other. And so... war, war never changes.

Even though a portion of humanity was to escape the terrible destructive power of the bombs themselves, the widespread extinction of terrestrial life and the poisoning of the air and water had serious ramifications for even the most sheltered of communities. Early in the year 2076 C.E., in preparation for the worst, the powers of the cloistered micronation of Gensokyo, located coterminously with what was then known as the State of Japan (a provisional member to the United States of America), decided to commission the \propername{Vault-Tec} company (a wholly owned subsidiary of Yakumo Industries) to build a series of underground shelters designed to safeguard its small population. Ten years after their internment, the ``Vaults'' were due to reopen, but when their inhabitants emerged they had only the hell of the wastes to greet them.

All, except those in \vaultnine. For on the fateful day when fire rained from the sky, the giant steel door of \vaultnine{} slid closed and never reopened... for over two hundred years. It was here you were born, and it is here you will die, because in \vaultnine, no one ever enters, and no one ever leaves.


\begin{commentary}
Canvassing for interest in a Fallout CYOA. Things seem slow enough here to put this here, but if anyone feels strongly about it, Anon can move it to \verb|/other/|. This would be a story written by Anonymous, so if the author gets hit by a bus, or if another Anon merely feels the updates are coming too slow, any other Anonymous might take the reins at any time (and if they decide to take up the challenge, Godspeed them). Anon has some of the character creation and introduction written out already, but hesitates to write more since votes tend to diverge from one's expectations. If you are interested in a post-apocalyptic CYOA but haven't played the game, I suggest you buy (or pirate) it for yourself as an early Christmas present, or at the very least check out the Fallout Wikia:

\verb|http://fallout.wikia.com/wiki/Fallout_Wiki|

I hope we'll meet again, my fellow boardsmen! But for now, we must part.

God Bless Touhou Project. God Bless Gensokyo.
\end{commentary}


\end{document}
